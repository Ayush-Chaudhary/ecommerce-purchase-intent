\documentclass[12pt]{article}
\usepackage[utf8]{inputenc}
\usepackage{graphicx}
\usepackage{amsmath}
\usepackage{hyperref}
\usepackage[margin=1in]{geometry}

\title{Predicting Purchase Intent in E-commerce}
\author{Ayush Chaudhary (120429125)}
\date{\today}

\begin{document}

\maketitle

\section{Introduction}
In the e-commerce industry, understanding and predicting user purchase intent is crucial for optimizing user experience and increasing conversion rates. I aim to develop a machine learning model that predicts the likelihood of a user purchasing a product they've viewed. This prediction can help e-commerce platforms implement targeted strategies to convert views into purchases.

\section{Dataset Overview}
I will utilize the Retail Rocket E-commerce Dataset \footnote{\url{https://www.kaggle.com/datasets/retailrocket/ecommerce-dataset}}, which contains 2,756,101 behavioral events (views, add-to-cart, transactions), 417,053 unique items with 20,275,902 property records, 1,407,580 unique visitors, and 1,669 category relationships. This rich dataset provides comprehensive information about user interactions and product characteristics.

\section{Proposed Approach}

\subsection{Target Variable}
I will create a binary classification model where the target variable indicates whether a user will purchase a product they've viewed (1) or not (0). This will be derived from the transaction events in the dataset, providing clear ground truth for model training and evaluation.

\subsection{Feature Engineering}
I will engineer features from three main categories. User behavior features will include the number of product views, time spent viewing, previous purchase history, add-to-cart events, and time between views. Product characteristics will encompass category information, price changes over time, availability status, and product properties. Temporal features will capture time of day/week of view, time since last purchase, and session duration.

\subsection{Algorithms}
I will evaluate multiple machine learning algorithms, starting with Logistic Regression as a baseline model. I will then implement and compare Random Forest, XGBoost, and Neural Networks to identify the most effective approach for this prediction task.

\section{Hypothesis}
I hypothesize that purchase intent can be predicted using user behavior patterns, product characteristics, and temporal features. Users who view a product multiple times are more likely to purchase, while products with stable prices tend to have higher conversion rates. Certain product categories consistently show higher purchase rates, and time-based patterns significantly influence purchase likelihood.

\section{Evaluation and Impact}
I will evaluate the model using AUC-ROC, Precision-Recall, F1 Score, and Confusion Matrix. The project's practical implications include improved conversion rates, better user experience, optimized marketing strategies, and enhanced inventory management. These improvements can lead to significant business value for e-commerce platforms.

% \section{Implementation Plan}
% I will complete the project in 6 weeks, with the first two weeks dedicated to data preprocessing and feature engineering. The following two weeks will focus on model development and training, while the fifth week will be spent on model evaluation and optimization. The final week will be dedicated to results analysis and documentation.

\end{document} 